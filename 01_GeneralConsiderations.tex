\section{Consideracoes Gerais}
Esta seção visa realizar um apanhado de considerações gerais sobre o
planejamento e execução do aplicativo de monitoramento do robô ROSA. Serão
abordados itens do processo de design, plano basico,
parâmetros de funcionamento, planejamento e avaliação de interação do
aplicativo.

\subsection{Processo de Design}
OProcesso de design do aplicativo tem ínicio na a análise inicial do
problema, onde são apontadas todas as situações que acarretam pro\-ble\-mas
durante as opera\-ções de inserção e remoção de stoplogs bem como o contexto de
usabilidade do produto. sao levadas em consideração
contexto de usabilidade, plano basico (wireframing) e parametros de fucionamento.

\begin{itemize}
  \item Laboratório-Aeroporto: carro particular;
  \item Aeroporto: despachados como bagagem pessoal;
  \item Aeroporto-Nova Mutum Paraná: carro alugado Hilux;
  \item Nova Mutum Paraná-Usina: carro alugado Hilux;
\end{itemize} 

Os três sensores indutivos, assim como o sonar, foram transportados em suas
respectivas caixas, em mochila particular durante todo o trajeto da viagem.


Podemos destacar algumas dificuldades de logística:
\begin{itemize}
  \item Transporte do case de $70 Kg$: apesar de apresentar rodinhas, o
  constante deslocamento do case entre carros era complicado e consumia tempo.
  \item O transporte de equipamentos com mais de 60kg deve ser realizado pelo
  serviços de transporte de cargas (ex: TAM Cargo), que além de ser o serviço
  específico para esse tipo de transporte, tem uma maior cobertura contra danos ocasionados pelo
  transporte.
  \item Não houve organização do pelican-case e do case KGB $70 Kg$. A
  necessidade de uma ferramenta ou equipamento tomava tempo pela busca e poderia
  até resultar na desmontagem do case para se ter acesso a um equipamento que
  estivesse no fundo.
  \item A falta de um carretel para o cabo umbilical dificultou bastante o seu
  manuseio, demandando muito tempo e esforço.
  \item Para o tubo, não foi projetado um case personalizado para transporte, o
  que resultou em uma proteção improvisada com espumas, e não facilitou a sua
  locomoção. Além de passar a impressão de desorganização e não profissionalismo
  ao cliente.

\end{itemize} 

\subsection{Planejamento}
A realização de um teste em campo exige o planejamento de dispositivos
necessários para sua execução levando em considração voltagem disponível na
USINA, equipamentos que podem ser fornecidos pela USINA, operários disponíveis
para realizar a operação, tempo de uso da viga e do ambiente. 

Sobre os equipamentos utilizados é importante que seja feita uma lista com todos
os equipamentos necessários e também a ordem que serão utilizados. Essa checagem
possibilita uma otimização do tempo de execução de cada teste e também reduz a
chance de esquecimento de algum ítem. Um ponto importante que pode ser
facilmente negligenciado é a contabilização das ferramentas necessárias para a
montagem e ajuste dos equipamentos. A montagem
e desmontagem das estruturas mecânicas não pode ser realizada de forma mais
eficiente pois havia somente uma chave canhão 8, impossibilitando a alocação de
mais de uma pessoa nessa tarefa.

Outro ponto importante é a listagem de toda a infraestrutura necessária para a
realização do teste que deverá ser fornecida pelo cliente ou responsável pelo
local. Ítens como fontes de energia e equipamentos cujo transporte não é
possível devem ser listados e solicitados com antecedência para uma preparação
eficiente. Todo o equipamento que possuir uma solução alternativa de backup
emergencial, deve ser considerado e levado a campo. Como exemplo é a utilização
de baterias como fonte de alimentação, não dependendo assim, de uma fonte de
alimentação no local. A carga e tempo de duração das baterias de laptops utilizados durante os
tete também deve ser levada em conta.

Por fim é ncessário também que se analise as condições climáticas e a estrutura
do local onde os testes serão realizados, onde os
equipamentos ficarão, se o local é coberto, qual tipo de clima, condições de poeira, para que nenhum equipamento
seja danificado e a equipe não seja exposta a nenhum risco desnecessário.
Em campo, o calor excessivo prejudicou bastante o primeiro
dia de testes, foi necessário a realização de uma pausa para que os computadores
esfriassem, já que ambos tinham travado por superaquecimento. A equipe também sofreu bastante, com dois membros passando mal
no final do dia. Como solução, uma tenda foi providenciada para os dias
seguintes.

\subsection{Montagem}

O planejamento da montagem dos equipamentos em campo é de suma importância pois
pode impossibilitar uma correta execução dos testes e fazer com que todo o
esforço desprendido para a realização dos mesmos seja desperdiçado.

É importante que o projeto da estrutura de acoplamento seja projetada a partir
do detalhamento técnico do equipamento presente em campo e do equipamento a ser
acoplado e deve possuir uma flexibilidade para adaptação devido a possíveis
imprevistos.
Foi verificado uma discrepância entre a viga pescadora e o seu modelo detalhado,
o que ocasionou em uma dificuldade de montagem. Entretanto, devido ao
planejamento de se projetar um suporte com múltiplos pontos de encaixe, foi possível realizar o acoplamento
em uma posição próxima à desejada a priori.
