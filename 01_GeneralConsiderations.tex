\section{Considerações Gerais}
Esta seção visa realizar um apanhado de considerações gerais sobre o
planejamento e execução da viagem. Serão abordados itens como logística de
materiais e dispositivos, planejamento e montagem.

\subsection{Logística de materiais e dispositivos}
Esta subseção abrange tanto pesquisa e escolha de malas/cases para
transporte de equipamentos e ferramentas, quanto organização interna dos
cases, proteção para transporte aéreo, meios de locomoção e dificuldades
encontradas.

A pesquisa e compra do pelican-case para transporte de ferramentas foi
realizada duas semanas antes da viagem, porém a alteração no escopo dos experimentos, na mesma
semana, exigiu uma nova pesquisa. Os professores Ramon e Jacoud avaliaram o
tempo ainda disponível e consideraram a viagem uma boa oportunidade para testar
os novos sensores que foram entregues:
profundímetro da Velki e sensor inercial (IMU), dos projetos LUMA e DORIS. Este
último faria o papel do inclinômetro no escopo do projeto. 

Devido ao teste extra, houve a necessidade de um novo projeto para a eletrônica
embarcada à prova d'água, cabos com emendas submarinas e montagem da estrutura mecânica, além da readaptação da placa eletrônica para os novos sensores com acréscimos de novos
CIs e uma grande reestruturação do software. A compra de novos componentes,
cases e cabos foram realizadas no centro da cidade do Rio de Janeiro, Rua
República do Líbano, pelo método de reembolso e com transporte particular.

O umbilical proposto para o novo teste da eletrônica é composto por um cabo
emborrachado de 12 vias com 40m de comprimento e um cabo Ethernet com 8 vias,
sem carretel.
O novo umbilical, somado com os outros diversos cabos da eletrônica, fontes,
baterias, voltímetro, osciloscópio, dentre outros equipamentos, exigiu a
aquisição de um novo case KGB com $65x65x65cm$ e rodas, totalizando um peso de
$70kg$ contando com o material.
Não foi possível realizar o planejamento e a construção de uma estrutura que
possibilitasse uma correta organização interna do case, já que o tempo da
última semana foi reservado, em sua maior parte, para a reestruturação da eletrônica e de software, a fim de garantir o último teste e a obtenção de dados.

O tubo à prova d'água que contém a eletrônica embarcada com profundímetro e
IMU foi enrolado em espuma, presa com abraçadeiras de plástico. No aeroporto,
foi necessário envolvê-lo com proteção adicional. A estrutura metálica do sensor
indutivo foi revestida com plástico bolha e também envolvido com proteção
adicional.

O transporte dos cases foram realizados do laboratório à Usina nas
seguintes etapas:
\begin{itemize}
  \item Laboratório-Aeroporto: carro particular;
  \item Aeroporto: despachados como bagagem pessoal;
  \item Aeroporto-Nova Mutum Paraná: carro alugado Hilux;
  \item Nova Mutum Paraná-Usina: carro alugado Hilux;
\end{itemize} 

Os três sensores indutivos, assim como o sonar, foram transportados em suas
respectivas caixas, em mochila particular durante todo o trajeto da viagem.


Podemos destacar algumas dificuldades de logística:
\begin{itemize}
  \item Transporte do case de $70 Kg$: apesar de apresentar rodinhas, o
  constante deslocamento do case entre carros era complicado e consumia tempo.
  \item O transporte de equipamentos com mais de 60kg deve ser realizado pelo
  serviços de transporte de cargas (ex: TAM Cargo), que além de ser o serviço
  específico para esse tipo de transporte, tem uma maior cobertura contra danos ocasionados pelo
  transporte.
  \item Não houve organização do pelican-case e do case KGB $70 Kg$. A
  necessidade de uma ferramenta ou equipamento tomava tempo pela busca e poderia
  até resultar na desmontagem do case para se ter acesso a um equipamento que
  estivesse no fundo.
  \item A falta de um carretel para o cabo umbilical dificultou bastante o seu
  manuseio, demandando muito tempo e esforço.
  \item Para o tubo, não foi projetado um case personalizado para transporte, o
  que resultou em uma proteção improvisada com espumas, e não facilitou a sua
  locomoção. Além de passar a impressão de desorganização e não profissionalismo
  ao cliente.

\end{itemize} 

\subsection{Planejamento}
A realização de um teste em campo exige o planejamento de dispositivos
necessários para sua execução levando em considração voltagem disponível na
USINA, equipamentos que podem ser fornecidos pela USINA, operários disponíveis
para realizar a operação, tempo de uso da viga e do ambiente. 

Sobre os equipamentos utilizados é importante que seja feita uma lista com todos
os equipamentos necessários e também a ordem que serão utilizados. Essa checagem
possibilita uma otimização do tempo de execução de cada teste e também reduz a
chance de esquecimento de algum ítem. Um ponto importante que pode ser
facilmente negligenciado é a contabilização das ferramentas necessárias para a
montagem e ajuste dos equipamentos. A montagem
e desmontagem das estruturas mecânicas não pode ser realizada de forma mais
eficiente pois havia somente uma chave canhão 8, impossibilitando a alocação de
mais de uma pessoa nessa tarefa.

Outro ponto importante é a listagem de toda a infraestrutura necessária para a
realização do teste que deverá ser fornecida pelo cliente ou responsável pelo
local. Ítens como fontes de energia e equipamentos cujo transporte não é
possível devem ser listados e solicitados com antecedência para uma preparação
eficiente. Todo o equipamento que possuir uma solução alternativa de backup
emergencial, deve ser considerado e levado a campo. Como exemplo é a utilização
de baterias como fonte de alimentação, não dependendo assim, de uma fonte de
alimentação no local. A carga e tempo de duração das baterias de laptops utilizados durante os
tete também deve ser levada em conta.

Por fim é ncessário também que se analise as condições climáticas e a estrutura
do local onde os testes serão realizados, onde os
equipamentos ficarão, se o local é coberto, qual tipo de clima, condições de poeira, para que nenhum equipamento
seja danificado e a equipe não seja exposta a nenhum risco desnecessário.
Em campo, o calor excessivo prejudicou bastante o primeiro
dia de testes, foi necessário a realização de uma pausa para que os computadores
esfriassem, já que ambos tinham travado por superaquecimento. A equipe também sofreu bastante, com dois membros passando mal
no final do dia. Como solução, uma tenda foi providenciada para os dias
seguintes.

\subsection{Montagem}

O planejamento da montagem dos equipamentos em campo é de suma importância pois
pode impossibilitar uma correta execução dos testes e fazer com que todo o
esforço desprendido para a realização dos mesmos seja desperdiçado.

É importante que o projeto da estrutura de acoplamento seja projetada a partir
do detalhamento técnico do equipamento presente em campo e do equipamento a ser
acoplado e deve possuir uma flexibilidade para adaptação devido a possíveis
imprevistos.
Foi verificado uma discrepância entre a viga pescadora e o seu modelo detalhado,
o que ocasionou em uma dificuldade de montagem. Entretanto, devido ao
planejamento de se projetar um suporte com múltiplos pontos de encaixe, foi possível realizar o acoplamento
em uma posição próxima à desejada a priori.
